\chapter{Implementation}

Containing a comprehensive description of the implementation of your software, including the language(s) and platform chosen, problems encountered, any changes made to the design as a result of the implementation, etc.

Ein wichtiges Konzept der räumlichen Statistik ist die räumliche Autokorrelation (engl. spatial autocorrelation).
Diese kann zum einen als Störfaktor gesehen werden, da statistische Tests verkompliziert werden. 
Bei räumlicher Autokorrelation liegt eine Abhängigkeit der Störgrößen des ökonometrischen Modells von regionalen 
Untersuchungseinheiten vor. Dies führt bei Kleinste-Quadrate-Schätzungen zu Verzerrungen der Regressionskoeffizienten 
oder Ungültigkeit der Signifikanztests. Zum anderen liefert diese zusätzliche Information Möglichkeiten 
zur räumlichen Interpolation. Während Trendprognosen eine zeitliche Autokorrelation benötigen, 
ermöglicht räumliche Autokorrelation (bzw. räumliche Persistenz) eine distanzabhängige Interpolation. 
In diesem Kapitel werden Eigenschaften und Berechnung der Statistiken untersucht.
