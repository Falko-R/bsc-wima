\chapter{Einführung} %engl. "Introduction"

A good proportion of the data out there in the real world is inherently spatial. 
From the population recorded in the national census, to every shop in your neighborhood, 
the majority of datasets have a location aspect that you can exploit to make the most of what they have to offer.

You will then combine different sources using their location as the bridge that puts them in relation to each other, allowing you to transform and repurpose them in different contexts.

Quantitative Revolution der Geographie?

Setting out the aims and objectives of your project, explaining the overall intention of the project and specific steps that will be taken to achieve that intention.

\section{Motivation}

Explaining the problem being solved.


\section{Aims and Objectives}

Ziele und Vorgaben der Arbeit.

Als Bachelorarbeit liegt die Absicht in einer möglichst breiten Aufgliederung vieler Aspekte, ohne (im Gegensatz zu einem Handbuch) Details und tiefere Analyse leisten zu können.  
Der Fokus liegt auf einer schnell und einfach lesbaren Aufbereitung der Grundlagen räumlicher Statistik mit einer groben Einteilung der größten Teilgebiete, 
sodass dem Leser eine schnelle und effiziente Übersicht über einen großen Forschungszweig zur eigenen Orientierung gegeben wird.

Zugleich wird die Anwendung auf ein spezielles Thema einen beispielhaften Einblick in den Prozess einer Modellbildung und die Pipeline der Datenanalyse 
zur Orientierung geliefert.

Dieser Spagat zwischen theoretischer Abstrahierung und praktischer Anwendung soll ein Gefühl der Anwendbarkeit und Relevanz vermitteln und die Bandbreite verdeutlichen.


\section{Description of the work}

Explaining what your project is meant to achieve, how it is meant to function, perhaps even a functional specification.


