\chapter{Einführung} %engl. "Introduction"


Ein Großteil aller uns umgebenden Informationen weist einen inhärent räumlichen Bezug auf, oft ohne dass wir diesen bewusst wahrnehmen.
Die räumliche und zeitliche Orientierung sind fundamentale Achsen der Informationsverarbeitung, 
von makroskopischen Sternenformationen über mikroskopische Proteinfaltungen hin zu unserer persönlichen Mobilität im Alltag. 
Wir nutzen diese Achsen, um alle weiteren Informationen in Beziehungen zu setzen, zu transformieren und im neuen Kontext zu verwenden. 
Diese Arbeit gibt einen Einblick in die wissenschaftliche Analyse räumlicher Daten 
zum Verständnis komplexer naturwissenschaftlicher und sozialer Phänomen.\\

Es werden in dieser Arbeit Methoden der \emph{raumdatenbezogenen Statistik} (engl. Spatial Statistics) untersucht und angewendet. 
Über die historischen Analysen mathematischer Zusammenhänge von Datenpunkten mit räumlichen Bezug zueinander wird in Kapitel \ref{ch:classification} berichtet.
Darüber hinaus hat die Anwendung auf reale Datensätze starken interdisziplinären Charakter. 
Der Forschungszweig der \emph{Spatial Analysis} im geografischen Kontext mit \emph{Geoinformationssystemen (GIS)} ist in dieser Hinsicht vergleichsweise jung. 
Die ersten bahnbrechenden Arbeiten stammen aus den späten 1950er Jahren und wurden grundlegend von der 
Entwicklung der Informationstechnologie (engl. \glqq Geography Quantitative Revolution\grqq) getrieben. 
Im Prinzip handelt es sich zu Beginn um die direkte Verknüpfung der Kartographie mit computergestützen Datenbanken.
Als wichtige Wegbereiter sind Roger F. Tomlinson am \emph{Canada Geographic Information System} sowie 
Howard T. Fisher als Gründer des \emph{Harvard Laboratory for Computer Graphics and Spatial Analysis} bekannt.

Die Überschneidungen der Geographie und Geologie, Biologie, Hydrographie, Meteorologie und 
vielen weiteren Fachgebieten begünstigte die Verbreitung dieser neuen Methodologie und Technologie. 
Weitere Anwendungsbeispiele unter Implementierung ähnliche Algorithmen und statistischer Methoden finden sich 
in der Ökologie (e.g. makroskopische Verteilungen von Lebensräumen, mikroskopische Proteinfaltungen), 
Soziologie (Entwicklung von Migrationsströmen), Anthropologie (Ausbreitungsmuster historischer Siedlungen),  
Astronomie und Kosmologie (Systemformationen und Galaxiencluster) sowie Neurologie (Auswertung von Gehirnscans). 
In der Informatik entwickelt der Bereich \emph{Spatial Data Mining} neue Algorithmen zur effizienten Verarbeitung großer 
Datenmengen mit Raumbezug und Nachbarschaftsbeziehungen.\\

Seit den frühen Tagen der Geoinformationstechnologie werden rasante Fortschritte erzielt.
Durch technische Errungenschaften wie etwa LIDAR, Dronen oder Formationen von Minisatteliten werden immense Ströme an Geodaten generiert.
Diese dienen etwa der landwirtschaftlichen Analyse von Feldern, der Sichtung illegaler Schifffahrts- oder Bebauungs- und Rodungsaktivitäten, 
der Fernüberwachung des Bewuchses von Bahntrassen und vielem mehr. 

Klassische Industrien und Einsatzgebiete zur Auswertung von georeferenzierten Daten umfassen nunmehr die Agrar- und Forstwirtschaft, 
Bergbau- und Förderindustrien, Luft- und Raumfahrt, Navigation, Nautik und maritime Wirtschaft, 
Sicherheitstechnik- und Rüstungsindustrie, Transport- und Logistik, Anlagen- und Immobilienerfassung sowie Kataster und Stadtplanung, 
Infrastruktur und Instandhaltung  von Schienen-, Telekommunikation-, Wasser-, Pipeline- und Stromnetzen, Katastrophenmanagement, 
Naturschutzgebiet- und Wildtierüberwachung und viele weitere. \\

Die Motivation dieser Bachelorarbeit liegt in einer möglichst breiten Aufgliederung vieler Aspekte, 
ohne umfangreiche Details und tiefere Analysen leisten zu können, wie sie etwa ein Handbuch bietet .  
Der Fokus liegt auf einer schnell und einfach lesbaren Aufbereitung der Grundlagen räumlicher Statistik mit einer groben Einteilung wichtiger Teilgebiete, 
sodass dem Leser eine schnelle und effiziente Übersicht über einen großen Forschungszweig zu seiner eigenen Orientierung gegeben wird.

Zugleich wird durch die praktische Anwendung auf lokale Geodaten ein beispielhafter Einblick 
in den Prozess der Modellbildung und die Pipeline der Datenanalyse geliefert. 
Dieser Spagat zwischen theoretischer Abstrahierung und praktischer Anwendung soll einen Eindruck der 
Anwendbarkeit und Relevanz vermitteln und die Bandbreite an Anwendungsmöglichkeiten verdeutlichen. 
Dies möge den Leser zu eigenen Projekten motivieren und ihm 
zusätzliche Werkzeuge und Fähigkeiten zur Analyse der immensen Datensätze zu liefern, welche inzwischen auch frei verfügbar sind.
