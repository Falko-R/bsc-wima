\chapter{Datenlage und GIS Konzepte}
\label{ch:background} 

Dieses Kapitel erläutert technische Voraussetzungen der Geoinformationssysteme (GIS) und ihrer räumlichen Datenanalyse, 
bevor im darauf folgenden Kapitel die Theorie eingeführt wird.

\section{GIS Software}

Der Markt für GIS-Anwendungen ist breit gefächert und stark segmentiert. 
Viele Anbieter sind auf Nischenlösungen spezialisiert und technische Neuerungen ändern die Marktverhältnisse fortwährend, 
was sowohl auf der Entwicklungs- als auch Nutzerseite solcher Software für ständigen Bedarf an Fachpersonal sorgt. 
Der seit den 80er Jahren etablierte ArcGIS-Anbieter ESRI sowie Open Source Alternativen wie QGIS, GRASS, und SAGA 
sind insbesondere in der akademischen Forschung und bei Behörden verbreitet. Dies betrifft zudem R und Python, welche durch Pakete bzw. Bibliotheken zu voll funktionsfähigen
GIS-Systemen erweiterbar sind sich hervorragend mit anderen Softwarelösungen integrieren lassen. Gegenüber klassischen relationalen Datenbanksystemen werden spezialisierte 
GIS-Datenbanken benötigt, welche mit den speziellen Vektorgeometrien und Rasterdaten operieren können. PostGIS für PostgreSQL ist ein Beispiel für diese Entwicklung.
Kommerzielle Lösungen wie Topobase und Map3D der Firma Autodesk, MicroStation von Bentley Systems sowie Geomedia von Intergraph finden 
vorwiegend als spezialisierte Industrielösungen Anwendung. 
Online-GIS mit Cloud-Anbindungen wie Google Maps, HERE und OpenStreetMap erweitern das Spektrum an Angeboten insbesondere für den privaten Endnutzer.\\

Die Daten dieser Arbeit wurden zunächst übersichtsweise mit MS Excel untersucht sowie nach relevanten Kriterien zusammengefasst und 
schließlich mittels der Open-Source Scriptsprache R weiterverarbeitet, einheitlich archiviert und schließlich numerisch modelliert und ausgewertet. 
Zwar ist der Umgang mit vielen Daten innerhalb von MS Excel zunächst intuitiver und interaktiver möglich, jedoch stößt die Software bei größeren Datensätzen schnell an Ihre Grenzen. 
Insbesondere ist die Analyse außerhalb klassischer „Spreadsheets“ nur bedingt möglich und bietet nicht genug Funktionalität für komplexe Datenanalysen. 
So exisitiert etwa keine Anbindung zur Auswertung geografischer Daten. Hierfür müsste etwa auf PowerBI, einer weiteren Software im Microsoft-Katalog zurückgegriffen werden.
Außerdem werden die Rohdaten nicht getrennt von der Modellierung und Analyse bearbeitet. 
Dies verletzt einen wichtigen Grundsatz für eine praktische und sichere Analyse in getrennten Dateien und verhält sich ähnlich zur Nutzung von LaTeX gegenüber Word.\\

R liefert eine etablierte Platform und mit dem zentralen \emph{cran}-Register ein sehr stabiles System an Bibliotheken.
Zu nennen sind insbesondere die Pakete \emph{dplyr} und \emph{ggplot2} aus dem Paketverbund \emph{tidyverse} von Hadley Wickham unter aktiven Entwicklung.
Derartige Pakete stellen einen großen Vorteil gegenüber vielen anderen Lösungen dar. Für die Raumdaten bieten sich \emph{raster} als Paket für Rasterdaten sowie \emph{sf} für Vektordaten an,
welches das etablierte Paket \emph{sp} ablöst. 
Einen guten Überblick bieten \cite[S. 53]{fischer_handbook_2010} sowie \cite{bivand_applied_2013}. 

\section{Projektionen und Koordinatensysteme}
\label{ch:projections}

Wichtige Grundbegriffe und Konzepte der raumdatenbezogenen Analyse sind die
räumlichen \emph{Merkmale} (engl. features), \emph{Träger} (engl. supports) 
und \emph{Attribute} (engl. attributes). \emph{Merkmale} bilden als Raumpunkte, Linien, Flächen und Volumen die Grundbausteine der Informationsverarbeitung. 
Räumliche \emph{Träger} definieren die genaue Größe, Form und Orientierung der Merkmale als zusätzliche Referenzinformation. 
Punkte etwa weisen weder Größe, Fläche noch Orientierung auf und besitzen den kleinsten Träger aller Merkmale. 
Regionen hingegen unterscheiden sich in allen drei Dimensionen voneinander.
\emph{Attribute} wiederum stellen Messwerte und Zufallsvariablen dar, welche mit den räumlichen \emph{Merkmalen} verknüpft werden.
Für eine genaue Übersicht und weitere Details wird auf \cite[S.38]{waller_applied_2004} verwiesen.\\

Das mathematische Teilgebiet der \emph{Geodäsie} behandelt die möglichst genaue Lokalisierung und Referenz
räumlicher Merkmale auf der Erdoberfläche und bedient sich hierfür fundamentaler Konzepte aus Geometrie und Topologie. 
Eine essentielle Grundlage bei der Arbeit mit georeferenzierten Daten bilden geographische und projizierte Koordinatensysteme.  

Ein \emph{geographisches Koordinatensystem} (engl. geographic coordinate system) ($\mathbf{CGS}$) 
definiert durch Längen- und Breitengrade ein perfektes sphärisches Koordinatensystem. 
Dieses Gradnetz (engl. graticule) lokalisiert Orte eindeutig auf einer Sphäre 
und mit Einschränkungen auf der imperfekten ellipsoiden Erdoberfläche.

Zur einfacheren Analyse und Darstellung werden diese dreidimensionalen, sphärischen Daten 
auf zweidimensionale Obeflächen wie Karten und Bildschirme projiziert. Diese Transformation 
geht zwangsweise mit Verzerrungen von Form, Flächeninhalt, Distanzen und Richtungen einher. 
Aus diesem Grund bestehen zahlreiche verschiedene Projektionen (e.g. Mercator, Albers) mit 
einer eigenen Balance der Verzerrungen. Als Ausgangspunkt der Transformation bestehen verschiedene
Standardellipsoide zur Approximation der Erdform. Die Orientierung und genaue Position 
des Ellipsoids relativ zur Erde wird als \emph{geodätisches Datum} bezeichnet. 
Ein lokales Datum approximiert dabei einen kleinen Teilbereich der Erdoberfläche besonders gut. 
In dieser Analyse verwenden wir den Standard \emph{WGS84} (World Geodetic System of 1984).
Nach der Projektion werden Vektoren, Polygone und Zentroide erzeugt. 
\cite[S. 49]{waller_applied_2004} enthält hierzu detaillierte Berechnungsvorschriften.

\section{Datengrundlage}

Genutzt werden öffentlich verfügbare Daten zu Bevölkerungsgrößen, Arbeitsmarktsituation, 
Flächeneigenschaften und weiteren Wirtschaftsdetails. 
Die \emph{shape}-Dateien zur Analyse von Vektordaten von Gemeinden und Bezirken 
sind durch das Geoportal Brandenburg und das OpenData Portal der Stadt Berlin frei verfügbar.

\subsection{Merkmale und Attribute}
Neben dieser Grundlage werden spezielle Daten des Unternehmensregisters über die Standorte von Unternehmen 
aus allen erfassten Wirtschaftsbranchen nach WZ-Systematik genutzt. 
Diese werden durch das \emph{Amt für Statistik Berlin-Brandenburg} öffentlich bereitgestellt.
Die Daten sind für Brandenburg auf Gemeindeebene und für Berlin auf Bezirkslevel aggregiert 
und liegen jahresweise zusammengefasst für einen Zeitraum von 2006 bis 2015 vor. 
Die Unternehmensstandorte des Registers liegen aus Datenschutzgründen nicht als punktgenaue GPS-Standorte oder Adressen vor, 
sondern geben nur die Gemeinde an, in welcher die Unternehmen angesiedelt sind. 
Auf diese räumliche Verdichtung der Daten und damit verbundene Probleme wird in Abschnitt \ref{ch:aggregation} genauer eingegangen. 
Weiterhin sind die Unternehmensgrößen nach Anzahl der Arbeitnehmer in 5 allgemeinen Größenklassen angegeben. 
Zur besseren Berechnung wurde hieraus ein ganzzahliger Schätzwert der Unternehmensgröße abgeleitet, 
um eine Beschränkung auf Methoden kategorischer Variablen zu vermeiden.  
Weitere Merkmale über die Unternehmensgröße wie etwa Umsatz oder Flächennutzung fehlen hierbei. 
Am problematischsten erweist sich jedoch die völlige Anonymisierung der einzelnen Unternehmen im Zeitberlauf. 
Es ist unbekannt, welche Merkmalsausprägung ein einzelnes Unternehmen in anderen Jahren aufweist, oder ob es dort überhaupt exisitert. 
Somit lassen sich einzelne Unternehmen weder örtlich noch zeitlich nachverfolgen und Aussagen über Standortwechsel können nur in allgemeiner Art und Weise getroffen werden.\\

Ein \emph{Unternehmen} oder \emph{Firma} ist gemäß des Thüringer Landesamts für Statistik eine rechtlich selbständige Einheit eines Gewerbebetriebs. 
Werden vom Unternehmen weitere selbständige, meist in anderen Orten befindliche Einheiten betrieben, 
ohne dass sie ein Tochterunternehmen sind, werden diese \emph{Zweigniederlassung} genannt.

In der Datenbank lassen sich sowohl Niederlassungen als auch Rechtliche Einheiten abrufen. 
Es geht in der folgenden Analyse um die Verteilung allgemeiner Standorte einer ganzen Branche im Raum und deren allgemeine zeitliche Entwicklung. 
Somit wird auf die Niederlassungen zugegriffen, welche zudem eine zahlenmäßig umfangreichere Datenbasis liefern.\\

Weitere Auswahlmöglichkeiten bestehen bei unterschiedlichen Tabellenformaten zur Speicherung der Informationen einzelner Gemeinden. 
Die Rohdaten wurden im für Datenbanken typischen LONG-Format heruntergeladen.  
Diese sind für Menschen unpraktisch und schlecht zu überschauen, 
für Programmfunktionen jedoch hervorragend geeignet und erlauben einen schnellen Datenabruf und effiziente Sortierung. 
Viele Merkmale werden jedoch redundant gespeichert und die Dateigröße wird somit aufgebläht. 
Datenbanken sind allerdings speziell für große Dateimengen ausgelegt und auf schnellen Abruf optimiert. 
Dieses Format wurde für die Analyse in R in ein WIDE-Format umgewandelt wie es für übersichtliche Spreadsheets und Tabellen üblich ist. 
Zudem passt es zu den Dataframes der Shapedateien und kann somit direkt verknüpft werden.

\subsection{Raumeinheiten}

Die räumlich feinste Gliederung bietet die sogenannte \glqq Raumabgrenzungen der Regionalstatistik auf Grenzen der administrativen Einheiten \grqq. 
In Brandenburg bezeichnet dies die Gemeinden und in Berlin die Bezirke. 
In Berlin existieren seit 2016 zudem Daten auf Ebene der Kieze, sogenannte \glqq Lebensweltlich orientierte Räume \grqq (LOR), 
als wichtigste kleinräumige Gliederung, deren räumliche Abgrenzung nach fachlichen Kriterien festgelegt wurde. \\

Derartige Grenzen werden aus administrativen und politischen Gründen im Zeitverlauf angepasst. 
Diese Änderungen müssen bei Betrachtungen über längere Zeiträume in die Shape-Datei integriert werden, um einheitliche Referenzen und Formate bei der Datenanalyse zu gewährleisten. 
Es handelt sich jedoch bisher in Brandenburg und Berlin ausschließlich um Eingemeindungen.
Damit treten nur Vergröberugen der Partition aus Gemeinden im Zeitverlauf auf, bzw. Verfeinerungen der Partition bei einem Rückblick in vergangene Jahre, was den Aufwand erheblich vereinfacht.
Beispiele für Eingemeindungen im Beobachtungszeitraum sind etwa die Integration von \glqq Hornow-Wadelsdorf\grqq{} in die Stadt Spremberg ab 2016 sowie die Eingemeindung von
\glqq Madlitz-Wilmersdorf\grqq{} nach \glqq Briesen (Mark)\grqq{} ab 2014.

%\section{Nichträumliche Analyse}

%Ein Ansatz der Modellierung ist die Analyse räumlich stochastischer \emph{Punktprozesse} (engl. Point Pattern Analysis). 
%Ein klassisches Beispiel ist der Poisson-Punkt-Prozess, mit Anwendungsfällen im Gesundheitswesen (Verbreitung von Epidemien) 
%in der Ökologie und Botanik (Wanderungsverhalten von Tieren, Verteilung von Pflanzenarten) und vielen mehr. 
%Wichtige Parameter sind die Intensität und… 
%Im vorliegenden Anwendungsfall ist jedoch problematisch, dass aufgrund der räumlichen Aggregation von konkreten Unternehmensstandorten auf die gesamte Gemeindeebene 
%kein kontinuierlich messbarer Raum vorliegt. Hier wurde behelfsmäßig mit Gemeindezentroiden gearbeitet, was zwar Ungenauigkeiten einführt, 
%aber insbesondere die spätere Kombination mit nachträglich recherchierten, exakten Standorten erlaubt. 

%Ein historischer Anwendungsfall, in welchem ebenfalls eine räumliche Aggregation eines realen Punktprozesses vorgenommen wurde, 
%ist die Einteilung Londons in Planquadrate nach dessen Bombardierung durch die deutsche Luftwaffe im 2. Weltkrieg, 
%um die Krater mit einer Poisson-Idealverteilung zu vergleichen und durch signifikante Abweichungen Hinweise auf die allgemeine Treffsicherheit der Bomber zu erhalten.

%Diese Methodik ist streng genommen nicht räumlich, denn die Daten werden aggregiert und in ihrer Dimension reduziert. 
%Es sollte grundsätzlich versucht werden, räumliche Daten so stark wie möglich zu vereinfachen und im besten Fall mit eindimensionalen Maßen zu arbeiten.
%Ein hervorragendes Beispiel \dots
%Erst danach sollte eine mehrdimensionale, räumliche Analyse stattfinden, welche ihre eigenen Probleme mit sich bringt, wie sie in Kapitel 3 beschrieben werden.
