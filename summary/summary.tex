\chapter{Zusammenfassung und Wertung}\label{ch:summary}


Es wurden die Clustereffekte zunächst nur für Daten aus 2014 untersucht. 
Im nächsten Schritt kann eine räumliche Zeitreihenanalyse folgen, 
um Konzentrationsvorgänge bestimmter Wirtschaftszweige wie der Logistik dynamisch zu beurteilen.

%Für die langjährige Analyse des gesamten Gebiets wurde auf Brandenburger Gemeinden und Berliner Bezirke zurückgegriffen. 
%Diese Teilgebiete weisen ähnliche Größenordnung in der Gesamtfläche auf.

Für detaillierte Analysen zum aktuellen Stand kann in Berlin auf die neue Einteilung nach LOR verfeinert werden, 
da Wirtschaft und Bevölkerung in den Berliner Bezirken vielfach stärker konzentriert ist als in Gemeinden.
Zudem muss ein adäquates Regressionsmodell formuliert werden, nachdem aussägekräftige Kovariablen identifiziert wurden.

\section{Anpassungen und Erweiterungen}

Um die Aussagekraft der Ergebnisse zu verbessern und auf zusätzliche Bereiche zu erweitern, 
werden zusätzliche Maßnahmen zur weiteren Verarbeitung empfohlen. Zur Untersuchung der Autokorrelation wurden mit Moran’s I und Geary’s c zwei klassische Maße verwendet.
Neben ihrer weiten Verbreitung und dem hohen Bekannheitsgrad profitiert der Anwender von den zahlreichen 
Untersuchungen der genauen Eigenschaften dieser Maße und ihrer statistischen Tests. 
Weitere Hilfsmittel wie der Autokorrelationstest von Kelejian-Robinson stehen zur Verfügung. 
Zur weiteren Untersuchung wird jedoch auch der Einsatz moderner Indizes empfohlen, um 
weitere Forschungsentwicklungen in die eigene Analyse zu integrieren. Wichtige Vertreter sind die G-Statistik von Getis und Ord \cite[Kapitel 10]{anselin_perspectives_2010}.
Auch die H-Statistik bietet weitere Vorteile.



%Weitere Anknüpfungspunkte 
%weitere Modelle welce die Erklärungskraft zusätzlich erhöhen und räumlcieh Daten trotz ihrer Komplexität sinnvolle Vorhersagen ihres Verhaltens erlauben.