\addcontentsline{toc}{chapter}{Abstract}



    \begin{quotation}
        \glqq […] \textit{Since all models are wrong the scientist cannot obtain a ``correct'' one by excessive elaboration. 
        On the contrary, following William of Occam he should seek an economical description of natural phenomena. 
        Just as the ability to devise simple but evocative models is the signature of the great scientist 
        so overelaboration and overparameterization is often the mark of mediocrity.}\grqq{} George E.P. Box, 1976
    \end{quotation}


\begin{abstract}

Es werden zunächst klassische Konzentrations- und Entropiemaße der deskriptiven räumlichen Statistik angewandt.
Darauf folgt die Anwendung einer Clusteranalyse durch globale Indizes (Moran’s I und Geary’s c)
und statistischer Tests auf Signifikanz, 
sowie spezialisierter lokaler Maße und Tests (Local Getis GI, Local Morans’ I). 
Daran knüpfen räumliche Regressionsmodelle an, deren Parameter durch Inferenzverfahren ermittelt werden.\\


\end{abstract}


\textbf{Erklärung}

Der Verfasser erklärt, dass er die vorliegende Arbeit selbständig, ohne fremde Hilfe und ohne Benutzung anderer 
als der angegebenen Hilfsmittel angefertigt hat. Die aus fremden Quellen (einschließlich elektronischer Quellen) 
direkt oder indirekt übernommenen Gedanken sind ausnahmslos als solche kenntlich gemacht. 
Wörtlich und inhaltlich verwendete Quellen wurden entsprechend den anerkannten Regeln wissenschaftlichen Arbeitens zitiert. 
Die Arbeit ist nicht in gleicher oder vergleichbarer Form, auch nicht  auszugsweise im Rahmen einer anderen Prüfung 
bei  einer anderen Hochschule vorgelegt worden.\\

Sie wurde bisher auch nicht veröffentlicht.\\

Ich erkläre mich damit einverstanden, dass die Arbeit mit Hilfe eines Plagiatserkennungsdienstes auf enthaltene Plagiate überprüft wird.\\
\\


Ort, Datum       $\qquad \qquad$           Unterschrift des Verfassers     